\addchap[Appendix B:$\quad$Proof of the Right-Hand Rule for the Cross Product]{Appendix B}
We will prove the right-hand rule\index{right-hand rule} for the cross product of two vectors in $\Real{3}$.\\

\par\noindent For any vectors \textbf{v} and \textbf{w} in $\Real{3}$, define a new vector,
$\textbf{n}(\textbf{v},\textbf{w})$, as follows:
\begin{enumerate}
 \item If \textbf{v} and \textbf{w} are nonzero and not parallel, and $\theta$ is the angle between them, then
  $\textbf{n}(\textbf{v},\textbf{w})$ is the vector in $\Real{3}$ such that:
  \begin{enumerate}[(a)]
   \item the magnitude of $\textbf{n}(\textbf{v},\textbf{w})$ is $\norm{\textbf{v}}\,\norm{\textbf{w}}\,\sin \theta$,
   \item $\textbf{n}(\textbf{v},\textbf{w})$ is perpendicular to the plane containing \textbf{v} and \textbf{w}, and
   \item \textbf{v}, \textbf{w}, $\textbf{n}(\textbf{v},\textbf{w})$ form a right-handed system.
  \end{enumerate}
 \item If \textbf{v} and \textbf{w} are nonzero and parallel, then $\textbf{n}(\textbf{v},\textbf{w}) = \textbf{0}$.
 \item If either \textbf{v} or \textbf{w} is $\textbf{0}$, then $\textbf{n}(\textbf{v},\textbf{w}) = \textbf{0}$.
\end{enumerate}
The goal is to show that $\textbf{n}(\textbf{v},\textbf{w}) = \Crossprod{\textbf{v}}{\textbf{w}}$ for all
\textbf{v}, \textbf{w} in $\Real{3}$, which would prove the right-hand rule for the cross product (by part 1(c) of
our definition). To do this, we will perform the following steps:\\

\par\noindent \emph{Step 1:} Show that $\textbf{n}(\textbf{v},\textbf{w}) = \Crossprod{\textbf{v}}{\textbf{w}}$ if
\textbf{v} and \textbf{w} are any two of the basis vectors \textbf{i}, \textbf{j}, \textbf{k}.\vspace{1mm}
\par\noindent This was already shown in Example \ref{exmp:crossijkfull} in Section 1.4.\enskip\checkmark\\

\par\noindent \emph{Step 2:} Show that $\textbf{n}(a\textbf{v},b\textbf{w}) = ab(\Crossprod{\textbf{v}}{\textbf{w}})$
for any scalars $a$, $b$ if \textbf{v} and \textbf{w} are any two of the basis vectors \textbf{i}, \textbf{j},
\textbf{k}.\vspace{1mm}
\par\noindent If either $a = 0$ or $b = 0$ then $\textbf{n}(a\textbf{v},b\textbf{w}) = \textbf{0} =
ab(\Crossprod{\textbf{v}}{\textbf{w}})$, so the result holds. So assume that $a \ne 0$ and $b \ne 0$. Let \textbf{v} and
\textbf{w} be any two of the basis vectors \textbf{i}, \textbf{j}, \textbf{k}. For example, we will show that the result
holds for $\textbf{v} = \textbf{i}$ and $\textbf{w} = \textbf{k}$ (the other possibilities follow in a similar
fashion).

For $a\textbf{v} = a\textbf{i}$ and $b\textbf{w} = b\textbf{k}$, the
angle $\theta$ between $a\textbf{v}$ and $b\textbf{w}$ is $90\Degrees$. Hence the magnitude of
$\textbf{n}(a\textbf{v},b\textbf{w})$, by definition, is
$\norm{a\textbf{i}}\,\norm{b\textbf{k}}\,\sin 90\Degrees = \abs{ab}$. Also, by definition,
$\textbf{n}(a\textbf{v},b\textbf{w})$ is perpendicular to the plane containing $a\textbf{i}$ and $b\textbf{k}$, namely,
the $xz$-plane. Thus, $\textbf{n}(a\textbf{v},b\textbf{w})$ must be a scalar multiple of \textbf{j}. Since its magnitude
is $\abs{ab}$, then $\textbf{n}(a\textbf{v},b\textbf{w})$ must be either $\abs{ab}\textbf{j}$ or $-\abs{ab}\textbf{j}$.

There are four possibilities for the combinations of signs for $a$ and $b$. We will consider the case when $a > 0$
and $b > 0$ (the other three possibilities are handled similarly).
\newpage
In this case, $\textbf{n}(a\textbf{v},b\textbf{w})$
must be either $ab\textbf{j}$ or $-ab\textbf{j}$. Now, since \textbf{i}, \textbf{j}, \textbf{k} form a
right-handed system, then \textbf{i}, \textbf{k}, \textbf{j} form a left-handed system, and so \textbf{i}, \textbf{k},
$-\textbf{j}$ form a right-handed system. Thus, $a\textbf{i}$, $b\textbf{k}$,
$-ab\textbf{j}$ form a right-handed
system (since $a > 0$, $b > 0$, and $ab > 0$). So since, by definition, $a\textbf{i}$, $b\textbf{k}$, 
$\textbf{n}(a\textbf{i},b\textbf{k})$ form a right-handed system, and since $\textbf{n}(a\textbf{i},b\textbf{k})$ has to
be either $ab\textbf{j}$ or $-ab\textbf{j}$, this means that we must have $\textbf{n}(a\textbf{i},b\textbf{k}) =
-ab\textbf{j}$.

But we know that $\Crossprod{a\textbf{i}}{b\textbf{k}} = ab (\Crossprod{\textbf{i}}{\textbf{k}}) =
ab (-\textbf{j}) = -ab\textbf{j}$. Therefore, $\textbf{n}(a\textbf{i},b\textbf{k}) =
ab(\Crossprod{\textbf{i}}{\textbf{k}})$, which is what we needed to show.\\
$\therefore ~~ \textbf{n}(a\textbf{v},b\textbf{w}) = ab(\Crossprod{\textbf{v}}{\textbf{w}}) \enskip\checkmark$\\

\par\noindent \emph{Step 3:} Show that $\textbf{n}(\textbf{u},\textbf{v} + \textbf{w}) =
\textbf{n}(\textbf{u},\textbf{v}) + \textbf{n}(\textbf{u},\textbf{w})$ for any vectors \textbf{u}, \textbf{v},
\textbf{w}.\vspace{1mm}
\par\noindent If $\textbf{u} = \textbf{0}$ then the result holds trivially since $\textbf{n}(\textbf{u},\textbf{v} +
\textbf{w})$, $\textbf{n}(\textbf{u},\textbf{v})$ and $\textbf{n}(\textbf{u},\textbf{w})$ are all the zero vector. If
$\textbf{v} = \textbf{0}$, then the result follows since $\textbf{n}(\textbf{u},\textbf{v} + \textbf{w}) =
\textbf{n}(\textbf{u},\textbf{0} + \textbf{w}) = \textbf{n}(\textbf{u},\textbf{w}) = \textbf{0} +
\textbf{n}(\textbf{u},\textbf{w}) = \textbf{n}(\textbf{u},\textbf{0}) = \textbf{n}(\textbf{u},\textbf{w}) =
\textbf{n}(\textbf{u},\textbf{v}) + \textbf{n}(\textbf{u},\textbf{w})$. A similar argument shows that the result holds
if $\textbf{w} = \textbf{0}$.

So now assume that \textbf{u}, \textbf{v} and \textbf{w} are all nonzero vectors.
We will describe a geometric construction of $\textbf{n}(\textbf{u},\textbf{v})$, which is shown in the
figure below. Let $P$ be a plane perpendicular to \textbf{u}. Multiply the vector \textbf{v} by the positive
scalar $\norm{\textbf{u}}$, then project the vector $\norm{\textbf{u}}\,\textbf{v}$ straight down onto the plane $P$.
You can think of this projection vector (denoted by $proj_{P} \norm{\textbf{u}}\,\textbf{v}$) as the shadow of the
vector $\norm{\textbf{u}}\,\textbf{v}$ on the plane $P$, with the light source directly overhead the terminal point of
$\norm{\textbf{u}}\,\textbf{v}$. If $\theta$ is the angle between \textbf{u} and \textbf{v}, then we see that
$proj_{P} \norm{\textbf{u}}\,\textbf{v}$ has magnitude
$\norm{\textbf{u}}\,\norm{\textbf{v}} \sin \theta$, which is the magnitude of $\textbf{n}(\textbf{u},\textbf{v})$. So
rotating $proj_{P} \norm{\textbf{u}}\,\textbf{v}$ by $90\Degrees$ in a counter-clockwise direction in the plane $P$
gives a vector whose magnitude is the same as that of $\textbf{n}(\textbf{u},\textbf{v})$ and which is perpendicular to
$proj_{P} \norm{\textbf{u}}\,\textbf{v}$ (and hence perpendicular to \textbf{v}). Since this vector is in $P$ then it is
also perpendicular to \textbf{u}. And we can see that \textbf{u}, \textbf{v} and this vector form a right-handed system.
Hence this vector must be $\textbf{n}(\textbf{u},\textbf{v})$. Note that this holds even if $\textbf{u} \parallel
\textbf{v}$, since in that case $\theta = 0\Degrees$ and so $\sin \theta = 0$ which means that
$\textbf{n}(\textbf{u},\textbf{v})$ has magnitude $0$, which is what we would expect.

\begin{center}
 \includegraphics{figappb.1.0}
\end{center}

Now apply this same geometric construction to get $\textbf{n}(\textbf{u},\textbf{w})$ and
$\textbf{n}(\textbf{u},\textbf{v} + \textbf{w})$. Since $\norm{\textbf{u}}\,(\textbf{v} + \textbf{w})$ is the sum of
the vectors $\norm{\textbf{u}}\,\textbf{v}$ and $\norm{\textbf{u}}\,\textbf{w}$, then the projection vector
$proj_{P} \norm{\textbf{u}}\,(\textbf{v} + \textbf{w})$ is the sum of the projection vectors
$proj_{P} \norm{\textbf{u}}\,\textbf{v}$ and $proj_{P} \norm{\textbf{u}}\,\textbf{w}$ (to see this, using the shadow
analogy again and the parallelogram rule for vector addition, think of how projecting a parallelogram onto a plane
gives you a parallelogram in that plane). So then rotating all three projection vectors by $90\Degrees$ in a
counter-clockwise direction in the plane $P$ preserves that sum (see the figure below), which means that
$\textbf{n}(\textbf{u},\textbf{v} + \textbf{w}) = \textbf{n}(\textbf{u},\textbf{v}) +
\textbf{n}(\textbf{u},\textbf{w}).\enskip\checkmark$\vspace{6mm}

\begin{center}
 \includegraphics{figappb.2.0}
\end{center}\vspace{6mm}

\par\noindent \emph{Step 4:} Show that $\textbf{n}(\textbf{w},\textbf{v}) = -\textbf{n}(\textbf{v},\textbf{w})$ for any
vectors \textbf{v}, \textbf{w}.\vspace{1mm}
\par\noindent If \textbf{v} and \textbf{w} are nonzero and parallel, or if either is $\textbf{0}$, then
$\textbf{n}(\textbf{w},\textbf{v}) = \textbf{0} = -\textbf{n}(\textbf{v},\textbf{w})$, so the result holds. So assume
that \textbf{v} and \textbf{w} are nonzero and not parallel. Then $\textbf{n}(\textbf{w},\textbf{v})$ has magnitude
$\norm{\textbf{w}}\,\norm{\textbf{v}}\,\sin \theta$, which is the same as the magnitude of
$\textbf{n}(\textbf{v},\textbf{w})$, and hence is the same as the magnitude of $-\textbf{n}(\textbf{v},\textbf{w})$.
By definition, $\textbf{n}(\textbf{v},\textbf{w})$ is perpendicular to the plane containing \textbf{w} and
\textbf{v}, and hence so is $-\textbf{n}(\textbf{v},\textbf{w})$. Also, \textbf{v}, \textbf{w},
$\textbf{n}(\textbf{v},\textbf{w})$ form a right-handed system, and so \textbf{w}, \textbf{v},
$\textbf{n}(\textbf{v},\textbf{w})$ form a left-handed system, and hence \textbf{w}, \textbf{v},
$-\textbf{n}(\textbf{v},\textbf{w})$ form a right-handed system. Thus, we have shown that
$-\textbf{n}(\textbf{v},\textbf{w})$ is a vector with the same magnitude as $\textbf{n}(\textbf{w},\textbf{v})$ and is
perpendicular to the plane containing \textbf{w} and \textbf{v}, and that \textbf{w}, \textbf{v},
$-\textbf{n}(\textbf{v},\textbf{w})$ form a right-handed system. So by definition this means that
$-\textbf{n}(\textbf{v},\textbf{w})$ must be $\textbf{n}(\textbf{w},\textbf{v}). \enskip\checkmark$\\

\par\noindent \emph{Step 5:} Show that $\textbf{n}(\textbf{v},\textbf{w}) = \Crossprod{\textbf{v}}{\textbf{w}}$ for all
vectors \textbf{v}, \textbf{w}.\vspace{1mm}
\par\noindent Write $\textbf{v} = \vecthreeijk{v}$ and $\textbf{w} = \vecthreeijk{w}$. Then by Steps 3 and 4, we have
\newpage
\begin{align*}
 \textbf{n}(\textbf{v},\textbf{w}) ~&=~ \textbf{n}(\vecthreeijk{v},\vecthreeijk{w})\\
 &=~ \textbf{n}(\vecthreeijk{v},\ssub{w}{1}\,\textbf{i}) ~+~ \textbf{n}(\vecthreeijk{v},\ssub{w}{2}\,\textbf{j} +
  \ssub{w}{3}\,\textbf{k})\\
 &=~ \textbf{n}(\vecthreeijk{v},\ssub{w}{1}\,\textbf{i}) ~+~ \textbf{n}(\vecthreeijk{v},\ssub{w}{2}\,\textbf{j}) ~+~
  \textbf{n}(\vecthreeijk{v},\ssub{w}{3}\,\textbf{k})\\
 &=~ -\textbf{n}(\ssub{w}{1}\,\textbf{i},\vecthreeijk{v}) ~+~ -\textbf{n}(\ssub{w}{2}\,\textbf{j},\vecthreeijk{v}) ~+~
  -\textbf{n}(\ssub{w}{3}\,\textbf{k},\vecthreeijk{v}) .
\end{align*}

We can use Steps 1 and 2 to evaluate the three terms on the right side of the last equation above:
\begin{align*}
 -\textbf{n}(\ssub{w}{1}\,\textbf{i},\vecthreeijk{v}) ~&=~ -\textbf{n}(\ssub{w}{1}\,\textbf{i},\ssub{v}{1}\,\textbf{i}) ~+~
 -\textbf{n}(\ssub{w}{1}\,\textbf{i},\ssub{v}{2}\,\textbf{j}) ~+~
 -\textbf{n}(\ssub{w}{1}\,\textbf{i},\ssub{v}{3}\,\textbf{k})\\
 &=~ -\ssub{v}{1}\ssub{w}{1}\textbf{n}(\textbf{i},\textbf{i}) ~+~ -\ssub{v}{2}\ssub{w}{1}\textbf{n}(\textbf{i},\textbf{j})
  ~+~ -\ssub{v}{3}\ssub{w}{1}\textbf{n}(\textbf{i},\textbf{k})\\
 &=~ -\ssub{v}{1}\ssub{w}{1}(\Crossprod{\textbf{i}}{\textbf{i}}) ~+~
    -\ssub{v}{2}\ssub{w}{1}(\Crossprod{\textbf{i}}{\textbf{j}}) ~+~
    -\ssub{v}{3}\ssub{w}{1}(\Crossprod{\textbf{i}}{\textbf{k}})\\
 &=~ -\ssub{v}{1}\ssub{w}{1}\textbf{0} ~+~ -\ssub{v}{2}\ssub{w}{1}\,\textbf{k} ~+~ -\ssub{v}{3}\ssub{w}{1}(-\textbf{j})\\
 -\textbf{n}(\ssub{w}{1}\,\textbf{i},\vecthreeijk{v}) ~&=~ -\ssub{v}{2}\ssub{w}{1}\,\textbf{k} ~+~
  \ssub{v}{3}\ssub{w}{1}\,\textbf{j}
\end{align*}
Similarly, we can calculate
\begin{align*}
 -\textbf{n}(\ssub{w}{2}\,\textbf{j},\vecthreeijk{v}) ~&=~ \ssub{v}{1}\ssub{w}{2}\,\textbf{k} ~-~
  \ssub{v}{3}\ssub{w}{2}\,\textbf{i}\\
 \intertext{and}
 -\textbf{n}(\ssub{w}{3}\,\textbf{j},\vecthreeijk{v}) ~&=~ -\ssub{v}{1}\ssub{w}{3}\,\textbf{j} ~+~
  \ssub{v}{2}\ssub{w}{3}\,\textbf{i} ~.
\end{align*}
Thus, putting it all together, we have
\begin{align*}
 \textbf{n}(\textbf{v},\textbf{w}) ~&=~ -\ssub{v}{2}\ssub{w}{1}\,\textbf{k} ~+~ \ssub{v}{3}\ssub{w}{1}\,\textbf{j} ~+~
  \ssub{v}{1}\ssub{w}{2}\,\textbf{k} ~-~ \ssub{v}{3}\ssub{w}{2}\,\textbf{i} ~-~ \ssub{v}{1}\ssub{w}{3}\,\textbf{j} ~+~
  \ssub{v}{2}\ssub{w}{3}\,\textbf{i}\\
 &=~ (\ssub{v}{2}\ssub{w}{3} - \ssub{v}{3}\ssub{w}{2})\textbf{i} ~+~ (\ssub{v}{3}\ssub{w}{1} -
  \ssub{v}{1}\ssub{w}{3})\textbf{j} ~+~ (\ssub{v}{1}\ssub{w}{2} - \ssub{v}{2}\ssub{w}{1})\textbf{k}\\
 &=~ \Crossprod{\textbf{v}}{\textbf{w}} \text{~~by definition of the cross product.}
\end{align*}
$\therefore ~~ \textbf{n}(\textbf{v},\textbf{w}) = \Crossprod{\textbf{v}}{\textbf{w}}$ for all vectors \textbf{v},
\textbf{w}.\enskip\checkmark\vspace{1mm}
\par\noindent So since \textbf{v}, \textbf{w}, $\textbf{n}(\textbf{v},\textbf{w})$ form a right-handed system, then
\textbf{v}, \textbf{w}, $\Crossprod{\textbf{v}}{\textbf{w}}$ form a right-handed system, which completes the proof.
