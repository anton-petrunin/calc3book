\addchap[Appendix A:$\quad$Answers and Hints to Selected Exercises]{Appendix A}
\textsf{\textbf{\Large Answers and Hints to Selected Exercises}}
\begin{multicols}{2}
\section*{Chapter \ref{chapter:vectors}}
\subsection*{Section \ref{sec1dot1} (p. \pageref{sec1dot1})}
\textbf{1.} (a) $\sqrt{5}$ \quad (b) $\sqrt{5}$ \quad (c) $\sqrt{17}$ \quad (d) $1$\\(e) $2 \sqrt{17}$ \quad \textbf{2.}
Yes \quad \textbf{3.} No
\subsection*{Section \ref{sec1dot2} (p. \pageref{sec1dot2})}
\textbf{1.} (a) $(-4,4,-3)$ \quad (b) $(2,6,-1)$\\(c) $\left ( \frac{-1}{\sqrt{30}},\frac{5}{\sqrt{30}},
\frac{-2}{\sqrt{30}}\right )$ \quad (d) $\frac{\sqrt{41}}{2}$ \quad (e) $\frac{\sqrt{41}}{2}$\\(f) $(14,-6,8)$ \quad
(g) $(-7,3,-4)$\\(h) $(-1,-6,1)$ \quad (i) $(-2,-4,2)$ \quad (j) No.\\\textbf{3.} No. $\norm{\textbf{v}} +
\norm{\textbf{w}}$ is larger.
\subsection*{Section \ref{sec1dot3} (p. \pageref{sec1dot3})}
\textbf{1.} $10$ \quad \textbf{3.} $73.4\Degrees$ \quad \textbf{5.} $90\Degrees$ \quad \textbf{7.} $0\Degrees$\\
\textbf{9.} Yes, since $\Dotprod{\textbf{v}}{\textbf{w}} = 0$.\\\textbf{11.} $\abs{\Dotprod{\textbf{v}}{\textbf{w}}} =
0 < \sqrt{21}\sqrt{5} = \norm{\textbf{v}}\,\norm{\textbf{w}}$\\\textbf{13.} $\norm{\textbf{v} + \textbf{w}} = \sqrt{26}
< \sqrt{21} + \sqrt{5} = \norm{\textbf{v}} + \norm{\textbf{w}}$\\\textbf{15.} Hint: use Definition
\ref{defn:dotprod}.\\
\textbf{24.} Hint: See Theorem \ref{thm:dotnorm}(c).
\subsection*{Section \ref{sec1dot4} (p. \pageref{sec1dot4})}
\textbf{1.} $(-5,-23,-24)$ \quad \textbf{3.} $(8,4,-5)$ \quad \textbf{5.} $\textbf{0}$\\\textbf{7.} $16.72$ \quad
\textbf{9.} $4\sqrt{5}$ \quad \textbf{11.} $9$ \quad \textbf{13.} $0$\\and $(8,-10,2)$ \quad \textbf{15.} $14$
\subsection*{Section \ref{sec1dot5} (p. \pageref{sec1dot5})}
\textbf{1.} (a) $(2,3,-2) + t(5,4,-3)$ \quad (b) $x = 2 + 5t,$\\$y = 3 + 4t, \, z = -2 - 3t$ \quad (c) $\frac{x - 2}{5} =
\frac{y - 3}{4} = \frac{z + 2}{-3}$\\\textbf{3.} (a) $(2,1,3) + t(1,0,1)$ \quad (b) $x = 2 + t,$\\$y = 1, \, z = 3 + t$
\quad (c) $x - 2 = z - 3, \, y = 1$\\\textbf{5.} $x = 1 + 2t, \, y = -2 + 7t, \, z = -3 + 8t$\\\textbf{7.} 7.65 \quad
\textbf{9.} $(1,2,3)$\\\textbf{11.} $4x - 4y + 3z - 10 = 0$\\\textbf{13.} $x - 2y - z + 2 = 0$\\\textbf{15.}
$11x - 24y + 21z - 26 = 0$ \quad \textbf{17.} $9/\sqrt{35}$\\\textbf{19.} $x = 5t$, $y = 2 + 3t$, $z = -7t$
\quad \textbf{21.} $(10,-2,1)$
\subsection*{Section \ref{sec1dot6} (p. \pageref{sec1dot6})}
\textbf{1.} radius: $1$, center: $(2,3,5)$ \quad \textbf{3.} radius: $5$, center: $(-1,-1,-1)$ \quad \textbf{5.} No
intersection.\\\textbf{7.} circle $x^2 + y^2 = 4$ in the planes $z = \pm \sqrt{5}$\\\textbf{9.} lines
$\frac{x}{a} = \frac{y}{b}$, $z = 0$ and $\frac{x}{a} = -\frac{y}{b}$, $z = 0$\\\textbf{13.}
$\left( \frac{2a}{2 - c}, \frac{2b}{2 - c}, 0 \right)$
\subsection*{Section \ref{sec1dot7} (p. \pageref{sec1dot7})}
\textbf{1.} (a) $(4,\frac{\pi}{3},-1)$ \quad (b) $(\sqrt{17},\frac{\pi}{3},1.816)$\\\textbf{3.} (a)
$(2\sqrt{7},\frac{11\pi}{6},0)$ \quad (b) $(2\sqrt{7},\frac{11\pi}{6},\frac{\pi}{2})$\\\textbf{5.} (a) $r^2 + z^2 = 25$
\quad (b) $\rho = 5$\\\textbf{7.} (a) $r^2 + 9z^2 = 36$ \quad (b) $\rho^2 ( 1 + 8 \cos^2 \phi ) = 36$\\
\textbf{10.} $(a,\theta,a \cot \phi )$ \quad \textbf{12.} Hint: Use the distance formula for Cartesian coordinates.

\section*{Chapter \ref{chapter:curves}}
\subsection*{Section \ref{sec1dot8} (p. \pageref{sec1dot8})}
\textbf{1.} $\textbf{f}\,'(t) = (1,2t,3t^2)$; $x = 1 + t$, $y = z = 1$\\\textbf{3.} $\textbf{f}\,'(t) =
(-2\sin 2t,2\cos 2t,1)$; $x = 1$,\\$y = 2t$, $z = t$ \quad \textbf{5.} $\textbf{v}(t) = (1,1 - \cos t,\sin t)$,\\
$\textbf{a}(t) = (0,\sin t,\cos t)$\\\textbf{9.} (a) Line parallel to \textbf{c} \quad (b) Half-line parallel to
\textbf{c} \quad (c) Hint:  Think of the\\functions as position vectors.\\\textbf{15.} Hint: Theorem
\ref{thm:vectripleprod}
\subsection*{Section \ref{sec1dot9} (p. \pageref{sec1dot9})}
\textbf{1.} $\frac{3\pi \sqrt{5}}{2}$ \quad \textbf{3.} $2 (5^{3/2} - 8)$ \quad \textbf{5.} Replace\\$t$ by
$\biggl( \biggl( \frac{27s + 16}{2} \biggr)^{2/3} - 4 \biggr)\bigg/ 9$ \quad \textbf{6.} Hint: Use\\Theorem
\ref{thm:vecdiffprops}(e), Example \ref{exmp:absvecderiv}, and\\Theorem \ref{thm:vectripleprod} \quad \textbf{7.} Hint:
Use Exercise 6.\\\textbf{9.} Hint: Use $\textbf{f}\,'(t) = \norm{\textbf{f}(t)}\textbf{T}$, differentiate that
to get $\textbf{f}\,''(t)$, put those expressions into $\Crossprod{\textbf{f}\,'(t)\,}{\,\textbf{f}\,''(t)}$, then
write $\textbf{T}\,'(t)$ in terms of $\textbf{N}(t)$. \quad \textbf{11.} $\textbf{T}(t) = \frac{1}{\sqrt{2}}
(-\sin t,\cos t,1)$, $\textbf{N}(t) = (-\cos t,-\sin t,0)$, $\textbf{B}(t) = \frac{1}{\sqrt{2}}(\sin t,-\cos t,1)$,
$\kappa(t) = 1/2$
\section*{Chapter \ref{chapter:mult-var-function}}
\subsection*{Section \ref{sec2dot1} (p. \pageref{sec2dot1})}
\textbf{1.} domain: $\Real{2}$, range: $\lbrack -1,\infty )$\quad
\textbf{3.} domain: $\lbrace (x,y): x^2 + y^2 \ge 4 \rbrace$, range: $\lbrack 0,\infty)$\\\textbf{5.} domain:
$\Real{3}$, range: $\ival{-1}{1}$\quad\textbf{7.} 1\\\textbf{9.} does not exist\quad
\textbf{11.} $2$\quad
\textbf{13.} $2$\quad
\textbf{15.} $0$\\\textbf{17.} does not exist
\subsection*{Section \ref{sec2dot2} (p. \pageref{sec2dot2})}
\textbf{1.} $\frac{\partial f}{\partial x} = 2x$, $\frac{\partial f}{\partial y} = 2y$\quad
\textbf{3.} $\frac{\partial f}{\partial x} = x(x^2 + y + 4)^{-1/2}$,
$\frac{\partial f}{\partial y} = \frac{1}{2}(x^2 + y + 4)^{-1/2}$\quad
\textbf{5.} $\frac{\partial f}{\partial x} = ye^{xy} + y$, $\frac{\partial f}{\partial y} = xe^{xy} + x$\quad
\textbf{7.} $\frac{\partial f}{\partial x} = 4x^3$, $\frac{\partial f}{\partial y} = 0$\\\textbf{9.}
$\frac{\partial f}{\partial x}=x(x^2 + y^2 )^{-1/2}$, $\frac{\partial f}{\partial y}=y(x^2 + y^2 )^{-1/2}$\\\textbf{11.}
$\frac{\partial f}{\partial x} = \frac{2x}{3}(x^2 + y + 4)^{-2/3}$,\\
$\frac{\partial f}{\partial y} = \frac{1}{3}(x^2 + y + 4)^{-2/3}$\quad\textbf{13.}
$\frac{\partial f}{\partial x}=-2xe^{-(x^2 + y^2 )}$,\\$\frac{\partial f}{\partial y}=-2ye^{-(x^2 + y^2 )}$\quad
\textbf{15.} $\frac{\partial f}{\partial x} = y\cos(xy)$,\\$\frac{\partial f}{\partial y} = x\cos(xy)$\quad
\textbf{17.}
$\frac{\partial^2 f}{\partial x^2} =2$,
$\frac{\partial^2 f}{\partial y^2} = 2$,\\
$\frac{\partial^2 f}{\partial x \,\partial y} = 0$\quad\textbf{19.}
$\frac{\partial^2 f}{\partial x^2} = (y+4)(x^2 + y+4)^{-3/2}$,\\
$\frac{\partial^2 f}{\partial y^2} = -\frac{1}{4}(x^2 + y+4)^{-3/2}$,\\
$\frac{\partial^2 f}{\partial x \,\partial y} = -\frac{1}{2}x(x^2 + y+4)^{-3/2}$\\\textbf{21.}
$\frac{\partial^2 f}{\partial x^2} = y^2 e^{xy}$,
$\frac{\partial^2 f}{\partial y^2} = x^2 e^{xy}$,\\
$\frac{\partial^2 f}{\partial x \,\partial y} = (1+xy)e^{xy} + 1$\quad
\textbf{23.} $\frac{\partial^2 f}{\partial x^2} =12x^2$,\\
$\frac{\partial^2 f}{\partial y^2} = 0$,
$\frac{\partial^2 f}{\partial x \,\partial y} = 0$\quad
\textbf{25.} $\frac{\partial^2 f}{\partial x^2} =-x^{-2}$,\\
$\frac{\partial^2 f}{\partial y^2} = -y^{-2}$,
$\frac{\partial^2 f}{\partial x \,\partial y} = 0$
\subsection*{Section \ref{sec2dot3} (p. \pageref{sec2dot3})}
\textbf{1.} $2x+3y-z-3=0$\quad
\textbf{3.} $-2x+y-z-2=0$\\\textbf{5.} $x+2y=z$\quad
\textbf{7.} $\frac{1}{2}(x-1)+\frac{4}{9}(y-2)+\frac{\sqrt{11}}{12}(z-\frac{2\sqrt{11}}{3})=0$\quad
\textbf{9.} $3x+4y-5z=0$
\subsection*{Section \ref{sec2dot4} (p. \pageref{sec2dot4})}
\textbf{1.} $(2x,2y)$\quad
\textbf{3.} $(\frac{x}{\sqrt{x^2 + y^2 + 4}},\frac{y}{\sqrt{x^2 + y^2 + 4}})$\\
\textbf{5.} $(1/x,1/y)$\quad
\textbf{7.} $(yz\cos(xyz),xz\cos(xyz),xy\cos(xyz))$\\
\textbf{9.} $(2x,2y,2z)$\quad
\textbf{11.} $2\sqrt{2}$\quad
\textbf{13.} $\frac{1}{\sqrt{3}}$\\
\textbf{15.} $\sqrt{3}\,\cos(1)$\quad\textbf{17.} increase: $(45,20)$,\\decrease: $(-45,-20)$
\subsection*{Section \ref{sec2dot5} (p. \pageref{sec2dot5})}
\textbf{1.} local min. $(1,0)$; saddle pt. $(-1,0)$\\\textbf{3.}
local min. $(1,1)$; local max. $(-1,-1)$; saddle pts. $(1,-1),(-1,1)$\quad
\textbf{5.} local min. $(1,-1)$; saddle pt. $(0,0)$\quad
\textbf{7.} local min. $(0,0)$\\\textbf{9.} local min. $(-1,1/2)$\quad\textbf{11.}
width = height = depth=$10$\quad\textbf{13.} $x=y=4$, $z=2$
\subsection*{Section \ref{sec2dot6} (p. \pageref{sec2dot6})}
\textbf{2.} $(\ssub{x}{0},\ssub{y}{0})=(0,0):$ $\rightarrow (0.2858,-0.3998)$;
$(\ssub{x}{0},\ssub{y}{0})=(1,1):$ $\rightarrow (1.03256,-1.94037)$
\subsection*{Section \ref{sec2dot7} (p. \pageref{sec2dot7})}
\textbf{1.} min. $\left(\frac{-4}{\sqrt{5}},\frac{-2}{\sqrt{5}}\right)$;
max. $\left(\frac{4}{\sqrt{5}},\frac{2}{\sqrt{5}}\right)$
\\
\textbf{3.}
min. $\left(\frac{20}{\sqrt{13}},\frac{30}{\sqrt{13}}\right)$;
max. $\left(-\frac{20}{\sqrt{13}},-\frac{30}{\sqrt{13}}\right)$
\\
\textbf{4.} There is no global maximum, nor global minimum.
\\
\textbf{5.} $\frac{8abc}{3\sqrt{3}}$
\section*{Chapter \ref{chapter:mult-integrals}}
\subsection*{Section \ref{sec3dot1} (p. \pageref{sec3dot1})}
\textbf{1.} $1$\quad
\textbf{3.} $\frac{7}{12}$\quad
\textbf{5.} $\frac{7}{6}$\quad
\textbf{7.} $5$\quad
\textbf{9.} $\frac{1}{2}$\quad
\textbf{11.} $15$
\subsection*{Section \ref{sec3dot2} (p. \pageref{sec3dot2})}
\textbf{1.} $1$\quad
\textbf{3.} $8\ln 2 - 3$\quad
\textbf{5.} $\frac{\pi}{4}$\quad
\textbf{6.} $\frac{1}{4}$\quad
\textbf{7.} $2$\quad
\textbf{9.} $\frac{1}{6}$\quad
\textbf{10.} $\frac{6}{5}$
\subsection*{Section \ref{sec3dot3} (p. \pageref{sec3dot3})}
\textbf{1.} $\frac{9}{2}$\quad
\textbf{3.} $(2\cos(\pi^2) + \pi^4 -2)/4$\quad
\textbf{5.} $\frac{1}{6}$\quad
\textbf{7.} $6$\\
\textbf{10.} $\frac{1}{3}$
\subsection*{Section \ref{sec3dot4} (p. \pageref{sec3dot4})}
\textbf{1.} The values should converge to $\approx 1.318$. (Hint: In Java the exponential function $e^x$ can be obtained
with \texttt{Math.exp(x)}. Other languages have
similar functions, otherwise use $e=2.7182818284590455$ in your program.)\\\textbf{2.} $\approx 1.146$\quad
\textbf{3.} $\approx 0.705$\quad
\textbf{4.} $\approx 0.168$
\subsection*{Section \ref{sec3dot5} (p. \pageref{sec3dot5})}
\textbf{1.} $8\pi$\quad
\textbf{3.} $\frac{4\pi}{3}(8-3^{3/2})$\quad
\textbf{7.} $1-\frac{\sin 2}{2}$\quad
\textbf{9.} $2\pi ab$
\subsection*{Section \ref{sec3dot6} (p. \pageref{sec3dot6})}
\textbf{1.} $(1,8/3)$\quad
\textbf{3.} $(0,\frac{4a}{3\pi})$\quad
\textbf{5.} $(0,3\pi/16)$\\\textbf{7.} $(0,0,5a/12)$\quad\textbf{9.} $(7/12,7/12,7/12)$
\subsection*{Section \ref{sec3dot7} (p. \pageref{sec3dot7})}
\textbf{1.} $\sqrt{\pi}$\quad
\textbf{2.} 1\quad
\textbf{6.} Both are $\frac{n}{(n+1)^2 (n+2)}$\quad
\textbf{7.} $\frac{1}{n}$\quad

\section*{Chapter \ref{chapter:surface-integrals}}
\subsection*{Section \ref{sec4dot1} (p. \pageref{sec4dot1})}
\textbf{1.} $1/2$\quad
\textbf{3.} $23$\quad
\textbf{5.} $24\pi$\quad
\textbf{7.} $-2\pi$\quad
\textbf{9.} $2\pi$\\\textbf{11.} $0$
\subsection*{Section \ref{sec4dot2} (p. \pageref{sec4dot2})}
\textbf{1.} $0$\quad
\textbf{3.} No\quad
\textbf{4.} Yes. $F(x,y)=\frac{x^2}{2}-\frac{y^2}{2}$\\
\textbf{5.} No\quad
\textbf{9.} (b) No. Hint: Think of how $F$ is defined.\quad
\textbf{10.} Yes. $F(x,y)=axy+bx+cy+d$
\subsection*{Section \ref{sec4dot3} (p. \pageref{sec4dot3})}
\textbf{1.} $16/15$\quad
\textbf{3.} $-5\pi$\quad
\textbf{5.} Yes. $F(x,y)=xy^2 + x^3$\quad
\textbf{7.} Yes. $F(x,y)=4x^2 y + 2y^2 + 3x$
\subsection*{Section \ref{sec4dot4} (p. \pageref{sec4dot4})}
\textbf{1.} $216\pi$\quad
\textbf{2.} $3$\quad
\textbf{3.} $12\pi/5$\quad
\textbf{7.} $15/4$
\subsection*{Section \ref{sec4dot5} (p. \pageref{sec4dot5})}
\textbf{1.} $2\sqrt{2}\,\pi^2$\quad
\textbf{2.} $(17\sqrt{17} - 5\sqrt{5})/3$\quad
\textbf{3.} $2/5$\\
\textbf{4.} $2$ \quad
\textbf{5.} $2\pi (\pi - 1)$ \quad
\textbf{7.} $67/15$ \quad
\textbf{9.} $6$\\
\textbf{11.} Yes \quad
\textbf{13.} No \quad
\textbf{19.} Hint: Think of how a vector field $\textbf{f}(x,y) = P(x,y)\,\textbf{i} + Q(x,y)\,\textbf{j}$ in
$\Real{2}$ can be extended in a natural way to be a vector field in $\Real{3}$.
\subsection*{Section \ref{sec4dot6} (p. \pageref{sec4dot6})}
\textbf{1.} $0$\quad
\textbf{3.} $12\sqrt{x^2 + y^2 + z^2}$\quad
\textbf{5.} $6(x+y+z)$\\
\textbf{7.} $12\rho$ \quad
\textbf{8.} $(4\rho^2 -6)e^{-\rho^2}$ \quad
\textbf{9.} $-\frac{2z}{r^3}\,\textbf{e}_{r} + \frac{1}{r^2}\,\textbf{e}_{z}$\\
\textbf{11.} $\text{div}~\textbf{f} = \frac{2}{\rho} - \frac{\sin\theta}{\sin\phi} + \cot\phi$;\\$\text{curl}~\textbf{f}
= \cot\phi\,\cos\theta\,\textbf{e}_{\rho} + 2\textbf{e}_{\theta} -2\cos\theta\,\textbf{e}_{\phi}$\\
\textbf{25.} Hint: Start by showing that $\textbf{e}_{r} = \cos\theta\,\textbf{i} + \sin\theta\,\textbf{j}$,
$\textbf{e}_{\theta} = -\sin\theta\,\textbf{i} + \cos\theta\,\textbf{j}$, $\textbf{e}_{z} = \textbf{k}$.
\end{multicols}
