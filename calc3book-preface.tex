\addchap{Preface}
This book covers calculus in two and three variables. It is suitable for a one-semester course, normally known as
``Vector Calculus'', ``Multivariable Calculus'', or simply ``Calculus III''. 
The prerequisites are the standard courses
in single-variable calculus (also known as Calculus I and II).

The exercises are divided into three categories: A, B and C. 
The A exercises are mostly
of a routine computational nature, the B exercises are slightly more involved, and the C exercises usually require
some effort or insight to solve. 
A crude way of describing A, B and C would be ``Easy'', ``Moderate'' and ``Challenging'', respectively. 
However, many of the B exercises are easy and not all
the C exercises are difficult.

Answers and hints to most odd-numbered and some even-numbered exercises are
provided in Appendix A. 

There are a few exercises that require the student to write a computer program,
for example, the Monte Carlo method for approximating multiple integrals, in
Section \ref{sec:Numerical Approximation of Multiple Integrals}.
The code samples in the text are in the Java programming language, hopefully with enough comments so that the reader can
figure out what is being done even without knowing Java. Those exercises do not mandate the use of Java, so
students are free to implement the solutions using the language of their choice. While it would have been simple to
use a scripting language like Python, and perhaps even easier with a functional programming language (such as Haskell or Scheme), 
Java was chosen due to its ubiquity, relatively clear syntax, and easy availability for multiple platforms.

This book is released under the GNU Free Documentation License (GFDL), which allows others to not only copy and
distribute the book but also to modify it. 
For more details, see the included copy of the GFDL. So that there is no
ambiguity on this matter, anyone can make as many copies of this book as desired and distribute it as desired,
without needing a permission.

This book can be downloaded at \url{https://github.com/anton-petrunin/calc3book};
the older, original version by Michael Corral,
can be also obtained from \url{http://www.mecmath.net}. 
