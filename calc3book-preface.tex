\addchap{Preface}
This book covers calculus in two and three variables. It is suitable for a one-semester course, normally known as
``Vector Calculus'', ``Multivariable Calculus'', or simply ``Calculus III''. The prerequisites are the standard courses
in single-variable calculus (a.k.a. Calculus I and II).

I have tried to be somewhat rigorous about proving results. But while it is important for students to
see full-blown proofs---since that is how mathematics works---too much rigor and emphasis on proofs
can impede the flow
of learning for the vast majority of the audience at this level. If I were to rate the level of rigor in the book on a
scale of 1 to 10, with 1 being completely informal and 10 being completely rigorous, I would rate it as a 5.

There are 420 exercises throughout the text, which in my experience are more than enough for a semester course in
this subject.
There are exercises at the end of each section, divided into three categories: A, B and C. The A exercises are mostly
of a routine computational nature, the B exercises are slightly more involved, and the C exercises usually require
some effort or insight to solve. A crude way of describing A, B and C would be ``Easy'', ``Moderate'' and ``Challenging'', respectively. 
However, many of the B exercises are easy and not all
the C exercises are difficult.

There are a few exercises that require the student to write his or her own computer program
to solve some numerical approximation problems (e.g. the Monte Carlo method for approximating multiple integrals, in
Section 3.4).
The code samples in the text are in the Java programming language, hopefully with enough comments so that the reader can
figure out what is being done even without knowing Java. Those exercises do not mandate the use of Java, so
students are free to implement the solutions using the language of their choice. While it would have been simple to
use a scripting language like Python, and perhaps even easier with a functional programming language (such as Haskell or Scheme), 
Java was chosen due to its ubiquity, relatively clear syntax, and easy availability for multiple platforms.

Answers and hints to most odd-numbered and some even-numbered exercises are
provided in Appendix A. Appendix B contains a proof of the right-hand rule for the cross product, which seems to have
virtually disappeared from calculus texts
over the last few decades. Appendix C contains a brief tutorial on Gnuplot for graphing functions of two variables.

This book is released under the GNU Free Documentation License (GFDL), which allows others to not only copy and
distribute the book but also to modify it. For more details, see the included copy of the GFDL. So that there is no
ambiguity on this matter, anyone can make as many copies of this book as desired and distribute it as desired,
without needing my permission.  The PDF version will always be freely available to the public at no cost
(go to \url{http://www.mecmath.net}). Feel free to
contact me at \texttt{\href{mailto:mcorral@schoolcraft.edu}{mcorral@schoolcraft.edu}} for any questions on this or any
other matter involving the book (e.g. comments, suggestions, corrections, etc). I welcome your input.

Finally, I would like to thank my students in Math 240 for being the guinea pigs for the initial draft of this book, and
for finding the numerous errors and typos it contained.

\begin{flushleft}
\emph{January 2008}\hspace{\stretch{1}}\textsc{Michael Corral}
\end{flushleft}
